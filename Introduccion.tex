\documentclass{beamer}

\usetheme{Berkeley}
\usecolortheme{whale}

\usepackage{booktabs}
\usepackage{tcolorbox}

\usepackage{color}
\usepackage[utf8]{inputenc}
\usepackage{pgf,tikz}
\usepackage{pgfplots}
\usepackage{hyperref}
\usepackage{ulem}
\usepackage{graphicx}

\hypersetup{pdfpagemode=FullScreen}
\pgfdeclareimage[height=1.0cm, width=1.0cm]{logo}{Logoucaribe.png}
\logo{\pgfuseimage{logo}}

\title{Sistemas de Información Geográfica}

\author{Héctor Fernando Gómez}
\institute{Universidad del Caribe\\ \tiny Cancún, México}
\date{Abril, 2018}

\begin{document}
\definecolor{lightgray}{rgb}{0.75,0.75,0.75}

\begin{frame}
\titlepage
\end{frame}

\begin{frame}{Outline}
\tableofcontents
\end{frame}

\AtBeginSection[]
{
  \begin{frame}
    \frametitle{Outline}
    \tableofcontents[currentsection]
  \end{frame}
}

\section{Sistema de información geográfica} 

\begin{frame}
	\frametitle{Algunas definiciones}		
	
	\begin{itemize}
		\item{Un conjunto de herramientas para el almacenamiento, la
recuperación, manipulación y análisis de datos espaciales (Marble 1984)}
		\item{Un sistema de apoyo a la toma de decisiones que permite
integrar datos georreferenciados en un ambiente de toma
de decisiones (Cowen 1988)}
		\item{Una actividad organizada por medio de la cual la gente
mide y representa fenómenos geográficos (Chrisman 1999)}
		\item{Es el conjunto de procesos, técnicas y tecnología que
permiten utilizar información espacial para analizar datos y
crear mapas (Freeman 2014)}
	\end{itemize}	
\end{frame}

\begin{frame}
	\frametitle{Aplicaciones}

	Algunas de las aplicaciones de los \textit{sistemas de información geográfica:}
	
	\begin{itemize}
		\item{\textbf{Manejo de recursos naturales:} un SIG permite cuantificar los recursos naturales disponibles en algún ecosistema en particular}
		\item{\textbf{Planeación y desarrollo urbano:} para la ubicación de instalaciones de servicio público como estaciones de bomberos y policías.}
		\item{\textbf{Transporte: } para la identificación de rutas óptimas para la distribución de bienes y servicios}
		\item{\textbf{Epidemiología:} un SIG puede permitir la identificación de factores ambientales que favorecen la difusión de diferenets enfermedades.}
	\end{itemize}
\end{frame}

\begin{frame}
	\frametitle{Contenido del curso}

	\begin{enumerate}
		\item{Datos espaciales}
		\item{Creación de conjuntos de datos espaciales}
		\item{Herramientas de geoproceso}
		\item{Análisis de terreno}
		\item{Procesamiento de imágenes satelitales}
		\item{Estadística espacial}
	\end{enumerate}			
	
\end{frame}

\begin{frame}
	\frametitle{Evaluación }

	\begin{enumerate}
		\item{Reporte de prácticas: \textbf{60} $\%$}
		\item{Análisis de artículos sobre aplicaciones de SIG: \textbf{40} $\%$}
	\end{enumerate}			
	
\end{frame}


\AtBeginSection[]
{
  \begin{frame}
    \frametitle{Outline}
    \tableofcontents[currentsection]
  \end{frame}
}

\section{Datos espaciales}

\begin{frame}
	\frametitle{Dato espacial}
	Un dato \textit{espacial} es una dupla compuesta de 
	
	\begin{enumerate}
		\item{Objeto espacial (\textbf{geométrico})}
		\item{Atributos (\textbf{variables de interés asociadas al objeto espacial})}
	\end{enumerate}
	
	$$\texttt{Dato espacial = (Objeto espacial, Atributos)}$$
		
\end{frame}


\begin{frame}
	\frametitle{Tipos de objetos espaciales}
	
	Existen cuatro tipos de datos espaciales 

	\begin{enumerate}
		\item{Puntos}
		\item{Líneas}
		\item{Polígonos}
		\item{Ráster}
	\end{enumerate}	
	
	En $\texttt{QGIS}$ generalmente estos datos espaciales se leen a partir de archivos con extensión $\texttt{shp}$.
	
\end{frame}

\begin{frame}
	\frametitle{Tipos de atributos}

	\begin{itemize}
		\item{Numéricos}
		\item{Categóricos}
	\end{itemize}		
	
\end{frame}



\begin{frame}
	\frametitle{Software}
	
	Algunas opciones de sistemas de información geográfica:

	\begin{itemize}
		\item{$\texttt{ArcGIS}$}	
		\item{$\texttt{GRASS}$}	
		\item{$\texttt{QGIS}$}	
	\end{itemize}
	
\end{frame}

\AtBeginSection[]
{
  \begin{frame}
    \frametitle{Outline}
    \tableofcontents[currentsection]
  \end{frame}
}

\section{Introducción a QGIS} 

\begin{frame}
	\frametitle{Elementos de la interfaz gráfica}
	
	\begin{itemize}
		\item{Canvas de mapas}
		\item{Explorador/Panel de capas}
		\item{Barra de herramientas}
		\item{Barra de estado}
	\end{itemize}
	
\end{frame}

\begin{frame}
	\frametitle{Apertura de archivos.}

	\begin{itemize}
		\item{Capa poligonal}
		\item{Capa de puntos}
		\item{Capa de líneas}
		\item{Capa ráster}
		\item{Mapa de contexto}
	\end{itemize}		
	
\end{frame}

\begin{frame}
	\frametitle{Tabla de atributos}

	\begin{center}
		\textbf{Tabla de atributos}
	\end{center}
	
\end{frame}

\end{document}

